\documentclass[answers]{exam}
\usepackage{graphicx}
\usepackage{wrapfig}
\usepackage[utf8]{inputenc}

\title{PSYCH 260.003}
\author{}
\date{April 7, 2017}

\pagestyle{headandfoot}
\firstpageheader{PSY 260}{Section 003}{Exam 3}
\runningheader{PSY 260}{Section 003}{Exam 3}
\firstpagefooter{}{Page \thepage\ of \numpages}{}
\runningfooter{}{Page \thepage\ of \numpages}{}

\begin{document}
\maketitle

\begin{center}
  \fbox{\fbox{\parbox{5.5in}{\centering
        Answer the questions using the Scantron form.}}}
\end{center}
\vspace{0.1in}
\makebox[\textwidth]{Name:\enspace\hrulefill}

\newpage

\section{Main}

\textbf{Questions 1 and 2 refer to the figure below.}

\begin{figure}[h]
\includegraphics[width=0.75\textwidth]{img/myotatic-reflex.jpg}
\centering
\end{figure}

\begin{questions}

% action
\question The figure depicts the \fillin, one of the simplest circuits in the nervous system. It regulates \fillin.
\begin{choices}
\choice biceptual reflex; balance.
\correctchoice myotatic/stretch reflex; muscle length/position.
\choice optokinetic reflex; muscle strength.
\choice Cartesian reflex; skeletal-muscular activity.
\end{choices}

% action
\question This circuit has a/an \fillin branch in which stretch receptors in intrafusal muscle fibers \fillin the extrafusal muscle fibers from the \emph{antagonist} muscle.
\begin{choices}
\correctchoice polysynaptic; inhibit.
\choice autonomic; inhibit.
\choice monosynaptic; excite.
\choice monosynaptic; inhibit.
\end{choices}

\newpage

% emotion
\question Plutchik's biological approach to emotion focuses on an emotion's \fillin and \fillin.
\begin{choices}
\choice subjective feelings; facial expressions.
\correctchoice action tendencies (tendency to approach or avoid); valence (positive/negative).
\choice intensity; subjective feelings.
\choice influence on on reproduction; link to defense behavior.
\end{choices}

% disorder
\question The serotonin hypothesis of depression suggests that \fillin.
\begin{choices}
\correctchoice lowered 5-HT levels are part of the disorder.
\choice increased 5-HT levels are part of the disorder.
\choice lower ACh levels exacerbate the disorder's positive symptoms.
\choice higher DA levels are linked to the disorder's positive symptoms.
\end{choices}

% action
\question The primary purpose of the \emph{extrafusal} muscle fibers is to \fillin.
\begin{choices}
\correctchoice Generate force.
\choice Inhibit the contraction of muscles.
\choice Sense tension/length.
\choice All of the above.
\end{choices}

% somatosensation
\question Which of the following is true regarding fibers that link somatosensory receptors to the central nervous system?
\begin{choices}
\choice Fibers that are smallest in diameter conduct information the fastest.
\choice Thin fibers are generally the most heavily myelinated.
\choice Temperature-related information is conducted faster than touch-related information.
\correctchoice Muscle spindle receptor axons are thickest and most heavily myelinated.
\end{choices}

% disorder
\question All of the following are treatments for bipolar disorder \emph{EXCEPT}:
\begin{choices}
\choice Lithium.
\choice Anticonvulsants.
\choice Antipsychotics.
\correctchoice Dopamine agonists.
\end{choices}

% somatosensation vision
\question One might be tempted to call the fingertips ``the somatosensory fovea'' for all of the following reasons \emph{EXCEPT}:
\begin{choices}
\choice Both the fingertips and the fovea have high receptor cell densities.
\choice Both the fingertips and the fovea have high perceptual acuity.
\correctchoice Both the fingertips and the fovea activate small areas of the cerebral cortex.
\choice The scanning movements of both the fingertips and the fovea are precisely controlled by the motor system.
\end{choices}

% action disorder
\question Parkinson’s Disease involves the degeneration of \fillin -releasing neurons in the \fillin.
\begin{choices}
\choice Acetylcholine; Thalamus.
\correctchoice Dopamine; Substantia Nigra.
\choice Dopamine; Inferior Colliculus.
\choice Acetylcholine; Substantia Nigra.
\end{choices}

\newpage

% action
\question The neurotransmitter \fillin is released by $\alpha$ motor neurons at the neuromuscular junction; this event leads to an \fillin within the muscle fiber and eventually, muscle fiber contraction.
\begin{choices}
\choice Glutamate; EPSP.
\choice Acetylcholine; IPSP.
\choice Glutamate; IPSP.
\correctchoice Acetylcholine; EPSP.
\end{choices}

% synaptic communication
\question Which of the following events must occur in order for neurotransmitter to be released from an axon's presynaptic terminal?
\begin{choices}
\choice Voltage-gated K+ channels must open to permit K+ to enter the cell.
\correctchoice Voltage-gated Ca++ channels must open to permit Ca++ to enter the cell.
\choice Neurotransmitters must diffuse through the cytoplasm to the presynaptic membrane.
\choice None of the above.
\end{choices}

% evolution
\question The human cerebellum is \fillin the rest of the brain when comparing it to related animal groups.
\begin{choices}
\choice larger than
\choice smaller than
\correctchoice the same size as
\choice less dense than
\end{choices}

\vspace{.25in}

\textbf{For the next three (3) questions match the correct label to the letters in the figure below.}

\begin{figure}[h]
\includegraphics[width=0.75\textwidth]{img/ventricles.jpg}
\centering
\end{figure}

% neuroanatomy
\question Third Ventricle
\question Lateral Ventricle
\question Fourth Ventricle

\newpage

% sleep
\question Visual information from the \fillin projects to the suprachiasmatic nucleus (SCN) of the \fillin. This is one way that light information influences circadian rhythms.
\begin{choices}
\correctchoice retina; hypothalamus.
\choice LGN; hippocampus.
\choice MGN; inferior colliculus.
\choice V1; thalamus.
\end{choices}

% development
% \question \fillin is a preventable (and treatable) birth defect characterized by a failure in the closure of \fillin neural tube.
% \begin{choices}
% \correctchoice Spina bifida; caudal
% \choice Anencephaly; caudal
% \choice Spina bifida; rostral
% \choice Anencephaly; rostral
% \end{choices}

% vision
% \question The anatomy of the retina seems ``backwards'' in that:
% \begin{choices}
% \choice Light receptive cells are positioned closest to the front of the eye.
% \choice Amacrine cells connect cells horizontally, while bipolar cells connect vertically.
% \correctchoice Light-receptive photoreceptors are positioned at the back of the eye.
% \choice The optic chiasm swaps information from the left and right visual fields.
% \end{choices}

% development
% \question In most areas of the human cerebral cortex, synaptic density peaks \fillin.
% \begin{choices}
% \choice when the neural tube closes
% \choice late in the fetal period
% \choice late in adult life
% \correctchoice in early to middle childhood
% \end{choices}

% sensory action
\question When you tap your eyeball, the world \fillin. This \fillin Dr. Wolpert's suggestion that the brain computes predictions about future sensory states.
\begin{choices}
\correctchoice seems to move; supports.
\choice remains still; supports.
\choice seems to move; undermines.
\choice remains still; undermines.
\end{choices}

% DTI/methods
\question Diffusion Tensor Imaging (DTI) is a/an \fillin MRI method that provides information about \fillin.
\begin{choices}
\choice functional; how neurotransmitters diffuse across the synaptic cleft
\choice functional; the blood oxygen-level dependent (BOLD) response
\correctchoice structural; connectivity between brain areas
\choice structural; the branching structure of neuronal dendrites
\end{choices}

% audition
% \question All of the following are components of the auditory projection from the cochlea to the cortex \emph{EXCEPT}:
% \begin{choices}
% \choice XIII (8th) cranial nerve.
% \correctchoice Lateral geniculate nucleus (LGN).
% \choice Inferior colliculus.
% \choice Superior olivary nucleus.
% \end{choices}

% disorder
\question Schizophrenia is characterized by which of the following brain abnormalities?
\begin{choices}
\choice Increased size of ventricles.
\choice Reduced hippocampal volume.
\choice Accelerated gray matter loss.
\correctchoice All of the above.
\end{choices}

% disorder
\question Why might the dopamine (DA) hypothesis not provide a comprehensive explanation for schizophrenia?
\begin{choices}
\choice Changes in DA levels have not been shown to disturb memory function.
\choice The hypothesis cannot explain the strong developmental origins of the disease.
\correctchoice Some drugs increase DA levels but reduce schizophrenic symptoms.
\choice DA antagonists only relieve the negative symptoms of schizophrenia.
\end{choices}

% disorder, glutamate
\question More recent theories about schizophrenia point to a disturbance in \fillin, one of the \fillin common neurotransmitters released in the CNS.
\begin{choices}
\choice endorphins; least
\choice norepinephrine; most
\choice CO; least
\correctchoice glutamate; most
\end{choices}

% development
% \question One of the \emph{last} events in the development of the nervous system is \fillin.
% \begin{choices}
% \choice The formation of the neural tube.
% \choice The differentiation of the pluripotent cells into neurons.
% \choice Synaptogenesis in the cerebral cortex.
% \correctchoice Myelination of cortical axons.
% \end{choices}

\newpage

% action disorder
\question Woody Guthrie and his mother died of \fillin, a disease that targets the \fillin.
\begin{choices}
\choice Parkinson’s Disease; basal ganglia.
\choice Parkinson’s Disease; cerebellum.
\correctchoice Huntington’s Disease; basal ganglia.
\choice Huntington’s Disease; cerebellum.
\end{choices}

% somatosensation
\question Touch receptors enervating the skin on the \fillin have especially \emph{small} receptive fields.
\begin{choices}
\correctchoice Face.
\choice Calf.
\choice Neck.
\choice Back.
\end{choices}

% somatosensation 5 vision 3
\question Perceptual sensitivity is \emph{NOT} related to which of the following?
\begin{choices}
\choice Receptor density.
\correctchoice Speed of propagation.
\choice Receptive field size.
\choice Size of the cortical area.
\end{choices}

% fear
\question In response to a typical environmental stressor, cortisol levels \fillin.
\begin{choices}
\choice involve activation of the SAM axis.
\choice rise, fall below baseline levels, then return.
\choice rise and stay elevated.
\correctchoice rise then return to normal after a short period.
\end{choices}

% \question It's hard to measure stress reactivity objectively in some animals (including humans) because
% \begin{choices}
% \choice We can't measure cortisol levels accurately
% \choice Animal behavior in stressful situations differs markedly from humans
% \correctchoice The \emph{perception} of stress can differ between individuals
% \choice Hypothalamic hormones have no influence on stress responses 
% \end{choices}

\question Which of the following statements about individuals with depression is NOT true?
\begin{choices}
\correctchoice Resting state studies have found a \emph{decrease} in connectivity between the orbitofrontal cortex (OFC) and other areas of the brain
\choice Activity levels between the amygdala and dorsolateral prefrontal cortex have been found to be inversely related
\choice Studies have found persistent activation in the amygdala
\choice Scientists have found an \emph{increase} in connectivity between the lateral prefrontal cortex and other areas
\end{choices}

% reward
% \question The \fillin in the ventral forebrain is the target of dopamine releasing neurons that originate in the ventral tegmental area of the \fillin; this forms a major pathway in the reward system.
% \begin{choices}
% \correctchoice nucleus accumbens; midbrain
% \choice dorsal striatum; thalamus
% \choice superior colliculus; hippocampus
% \choice amygdala; hypothalamus

% vision 4
% \question Outside the fovea, the retina contains \fillin but has \fillin.
% \begin{choices}
% \correctchoice More rod photoreceptors; lower visual acuity.
% \choice Fewer rod photoreceptors; greater visual acuity.
% \choice More cone photoreceptors; fewer total ganglion cells.
% \choice Fewer cone photoreceptors; more ganglion cells.
% \end{choices}

% vision 5
% \question If long wavelength cones respond best to lights which we perceive as red, \fillin wavelength cones respond best to \fillin light.
% \begin{choices}
% \choice Short; green.
% \choice Medium; yellow.
% \correctchoice Short; blue.
% \choice Medium; blue.
% \end{choices}

\newpage

\textbf{Indicate the letter of the lobe that corresponds to the location of each sensory or motor cortical area.}

\begin{figure}[h]
\includegraphics[width=0.75\textwidth]{img/lobes.jpg}
\centering
\end{figure}

% somatosensation 7 vision 10 audition 1
\question Location of the primary somatosensory cortex.
\question Location of the primary auditory cortex.
\question Location of the primary motor cortex.
\question Location of the primary visual cortex.

\vspace{.25in}

\textbf{Select the best answer for the following questions.}

% action 9 hormones 1
\question The uterus consists of \fillin muscle fibers that contract involuntarily in the presence of the hormone \fillin.
\begin{choices}
\choice Striated; cortisol.
\choice Striated; oxytocin.
\correctchoice Smooth; oxytocin.
\choice Smooth; melatonin.
\end{choices}

\newpage

% vision 11 audition 2
% \question Which two sensory streams provide the most precise information about objects or animals at a distance – distal to the observer?
% \begin{choices}
% \correctchoice vision and audition.
% \choice somatosensation and gustation.
% \choice olfaction and the temperature sense.
% \choice vestibular sense and vision.
% \end{choices}

% sensory systems
\question The \fillin of your smartphone is analogous to the pressure receptors in your skin.
\begin{choices}
\choice accelerometer
\choice cellular radio
\correctchoice touch screen
\choice GPS transmitter
\end{choices}

% disorder
\question Cognitive behavior therapy is \emph{less} successful than drugs in treating depression.
\begin{choices}
\choice True.
\correctchoice False.
\end{choices}

% reward system
\question Milner and Olds discovered that electrical stimulation of the medial forebrain bundle connecting the \fillin and \fillin caused experimental animals to change their behavior in order to seek out ever more frequent stimulation.
\begin{choices}
\correctchoice ventral tegmental area; nucleus accumbens
\choice hippocampus; amygdala
\choice temporal cortex; striatum
\choice hypothalamus; pituitary
\end{choices}

% vision
% \question 90\% of projections from the retina go to which brain region?
% \begin{choices}
% \choice Ventral lateral thalamus.
% \correctchoice Lateral Geniculate Nucleus.
% \choice Somatosensory cortex.
% \choice Hypothalamus.
% \end{choices}

% neurotransmitters
\question Serotonin/Norepinephrine reuptake inhibitors (SNRIs) act on presynaptic \fillin and cause extracellular levels of these \fillin to be increased.
\begin{choices}
\choice metabotropic receptors; hormones.
\choice ion pumps; amino acids.
\correctchoice transporters; monoamines.
\choice Ionotropic receptors; indolamines.
\end{choices}

% thermo/chemo receptors
\question Spicy foods can seem `hot' even at room temperature because \fillin.
\begin{choices}
\choice thermoreceptors in the skin don't respond to temperature differences
\correctchoice thermoreceptors in the skin also respond to certain chemical substances
\choice flavor involves the olfactory system and the gustatory system
\choice receptive fields for temperature overlap with those for flavor
\end{choices}

% vision
% \question The fiber crossing at the optic chiasm serves what purpose?
% \begin{choices}
% \choice Auditory and visual information end up on the same side of the brain.
% \choice Visual information from the left eye projects to the right brain and vice versa.
% \choice Visual information originating from the right side of space projects to the same hemisphere, and vice versa.
% \correctchoice Visual information originating from the left side of space projects to the opposite hemisphere, and vice versa.
% \end{choices}

% somatosensation  action
\question Elephants have high levels of dexterity (fine motor control) in their trunks. Somatosensory neurons in the trunk region of the elephant`s S1 are likely to have \fillin.
\begin{choices}
\correctchoice Small receptive fields.
\choice Large receptive fields.
\choice Weak projections to corresponding regions of M1.
\choice Low levels of myelination.
\end{choices}

% action disorder
\question Which of these is an effective treatment of Huntington’s Disease?
\begin{choices}
\choice Dopamine Agonists
\choice NMDA Agonist
\choice Selective Serotonin Reuptake Inhibitors
\correctchoice None of the above
\end{choices}

\newpage

% receptive field
\question A somatosensory neuron's receptive field consists of \fillin.
\begin{choices}
\choice the skin between cutaneous receptor dendrites
\correctchoice the region of the skin that influences the neuron's firing when stimulated
\choice all the inputs to the neuron's dendrites and soma
\choice its response pattern to `donut'-shaped inputs
\end{choices}

% vision
% \question The CNS compares auditory signals between the two ears in order to calculate \fillin.
% \begin{choices}
% \choice distance to an auditory target.
% \choice the shape or form of an auditory target.
% \correctchoice the left/right position of an auditory target.
% \choice the timbre of an auditory target.
% \end{choices}

% learned fear
\question Lesions of the \fillin block fear conditioning in experimental animals.
\begin{choices}
\choice hippocampus
\choice cerebral cortex
\correctchoice amygdala
\choice striatum
\end{choices}

\vspace{2in}
\begin{center}
\textbf{Turn the page to answer the bonus questions}.
\end{center}

\newpage
\section{Bonus}

% vision
% \question Why does Gilmore think the retina is ``physiologically backwards''?
% \begin{choices}
% \correctchoice Light hyperpolarizes photoreceptors, decreasing neurotransmitter release.
% \choice Light depolarizes ganglion cells; increasing neurotransmitter release.
% \choice Different colors of light change a photoreceptor’s resting potential in similar ways.
% \choice Photoreceptors respond both to chemical and thermal signals.
% \end{choices}

% disorder
\question Which of these is \emph{NOT} true about individuals with schizophrenia?
\begin{choices}
\correctchoice About half of them have a moderate form that is manageable.
\choice About a third of them have a mild form that resolves.
\choice They show decreased cortical thickness in adolescence.
\choice They can exhibit delusional thoughts, hallucinations, mood issues, and behavioral abnormalities.
\end{choices}

% emotion
\question The projection from the \fillin to the \fillin is a major pathway in the brain's `reward' system.
\begin{choices}
\correctchoice Ventral tegmental area (VTA); nucleus accumbens/ventral striatum.
\choice Substantia nigra; striatum.
\choice Ventral tegmental area (VTA); amygdala.
\choice Hypothalamus; adrenal medulla.
\end{choices}

\question What did Dr. Wolpert say the sea squirt does after it finds a home on a rock?
\begin{choices}
\choice Looks for something to eat.
\correctchoice Eats its own brain.
\choice Rests and digests.
\choice Starts seeking a mate.
\end{choices}

% stress
\question An acute stressor is one that \fillin.
\begin{choices}
\correctchoice lasts only a short period of time
\choice is especially intense and long-lived
\choice rarely triggers the HPA axis
\choice overstimulates cortisol receptors in the spinal cord
\end{choices}

\end{questions}
\end{document}
